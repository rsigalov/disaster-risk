\documentclass[11pt]{article}
\usepackage[utf8]{inputenc}

\title{}
\author{}
\date{\today}

\usepackage{amsmath, amsfonts, amsthm, amssymb}
% \usepackage{fourier} 

\usepackage[colorlinks=true, linkcolor=blue, urlcolor=blue, citecolor = blue]{hyperref}

\usepackage[margin = 0.9in]{geometry}
% \usepackage{xcolor}
\usepackage{listings}
\usepackage{color} %red, green, blue, yellow, cyan, magenta, black, white
\definecolor{mygreen}{RGB}{28,172,0} % color values Red, Green, Blue
\definecolor{mylilas}{RGB}{170,55,241}
% \usepackage[framed,numbered,autolinebreaks,useliterate]{mcode}

\usepackage{graphicx}
\usepackage{subcaption}
\usepackage[normalem]{ulem}

\newcommand{\parder}[2]{\frac{\partial #1}{\partial #2}}
\newcommand{\ex}{\mathbb{E}}
\newcommand{\eps}{\varepsilon}

\usepackage[flushleft]{threeparttable}

\usepackage{booktabs}
\usepackage{setspace}
% \spacing{1.3}

\newtheorem{proposition}{Proposition}


\begin{document}

\begin{enumerate}
	\item $D_{it}$ is a not a measure of disasters but picks up any non-diffusion part of the return process
	\begin{itemize}
		\item R1 asks how to distinguish between a process with many small jumps and a disaster
		\item R1 points out that a set of paper show that option are well decribed by a pure jump process. \textit{Trying to get on process of the stock price and not describe option dynamics}
		\item R1: Lee and Mykland (2008) and Lee (2011) show that stock prices jump on earnings announcements and FED announcements $\Rightarrow$ such jumps may be priced in the options as opposed to disaster risk. This becomes more complicated if news arrival is clustered in time. \textit{Should be possible to compare options at maturities that do and don't include such examples of jumps}
		\item R4 is also worried about this.
	\end{itemize}
	\textbf{Comment:} We can deal with this concern by estimating $D_{it}$ for out of the money options (strike $<$ $P_0 - 2\sigma$ for example) as opposed to whole the whole set of strikes.

	\item Relate to and contrast with different explanations for similar implications of other models
	\begin{itemize}
		\item R1: The result that $p^*$ predicts macro variables don't support disaster model per se. The same predictive power can be generated in a production model with time-varying risk premia. \textit{Can identify variables that change only during disasters and see if $p_t^*$ predicts them}
		\item R4: focus on one type of evidence to make a strong case that it is consistent with rare disasters and not other explanations
	\end{itemize}

	\textbf{Comment:} These concerns should go away after we deal with the previous point

	\item R4 suggests to put more emphasis on cross-sectional results: does the factor help to explain other anomalies, e.g Fama-French, Profitability, Asset Growth and others

	\textbf{Comment:} Will take this suggestion seriously after the main estimation

	\item Study return predictablity based on $p^*_t$ as a direct prediction of the model. 
	\begin{itemize}
		\item R1 points to Anderson, Fusari and Todorov (2015), Bollerslev, Todorov and Xu (forthcoming) and Bollerslev, Tauchen and Zhou (2009) as studying return predictability due to jump risk
	\end{itemize}
	\textbf{Comment:} Not a big deal, easy to include results after the main estimation.
	
	\item R1 points that if the return process has both common and idiosyncratic components, then even if idiosyncratic jumps are uncorrelated, averaging of $D_{it}$ across $i$ picks up both the probability of a common jump and something like cross-sectional variance of idiosyncratic components.
	\begin{itemize}
		\item R2 on the other side says that the fact that $p^*_{t}$ is the same across stocks is obvious and suggests to put less emphasis on theory behind it
	\end{itemize}
	\textbf{Comment:} Emil thinks that this is not very relevant. To alleviate concerns we can simulate returns with both aggregate and idiosyncratic jump components and show that idiosyncratic component doesn't affect the results.

	\item Suggest to start from index options
	\begin{itemize}
		\item R3: Can compute the entire risk-neutral distribution from index options given sufficient number of strike prices. Probably provides a more accurate measure of disaster risk
		\item R4: if all stocks jump simultaneously index options should be sufficient. If these is a systematic component why the correlation between the factor and an index jump risk is as small as 69\%.
	\end{itemize}

	\textbf{Comment:} Can provide these results as well, not a big problem once we deal with the estimation.

	\item Related papers
	\begin{itemize}
		\item R2: Christoffersen, Fournier and Jacobs (R\&R in RFS) that does a similar exercise.
		\item R2: since $p^*_t$ is very correlated with VIX worry about the relation with Ang, Hodrick, Xing and Zhang (JF, 2006)
		\item R3: Cremers et al. (JF, 2015)
		\item R1 points to Gao and Song (2015) that studies the same $D_{it}$ but uses it directly to explain differences in returns as opposed to constructing portfolios. \textbf{Comment:} Using measure $D_{it}$ directly results in too few observations because not so many stocks have liquid options.
		\item R4: Why is Du and Kapadia (2013) method better than Bollerslev, Todorov and Xu variance premium and Kelley and Jiang extreme stock returns (\textit{I think both of these develop an ex post measure})
	\end{itemize}

	\textbf{Comment:} The majority of these papers develops new approaches to estimate jump component and of course all of them are correlated. However, we are inherently interested in what can we learn from the resulting measure in terms of time series: does $p_t^*$ move too much and in cross section: is the exposure to $p_t^*$ priced in the cross section.

	\item Clarification of the model and the estimation process
	\begin{itemize}
		\item R1: is PCA conducted on a balanced panel? Are variables standardized before doing PCA? \textit{I think in PCA you always standardize the variables}
		\item R3 asks for a less obscure model to see the direct effect of rare disasters as opposed to something else.
		\item R4 asks for clarification on Du and Kapadia (2013) method: why does it work intuitively and measures what the paper wants to measure. What are the assumptions behind this method?
	\end{itemize}

	\item Numbers
	\begin{itemize}
		\item R2: Surprised that the first PC is less than 50\%
		\item R2: Annualized AR(1) is only 30\% which is not very persistent. \textbf{Comment:} this may be good if we want to argue that $p_t^*$ moves too much. Can also look at the turnover in the sorted portfolios to make the same point.
		\item R3: relate 50\% of the first PC to some benchmark to get some idea about relative magnitudes.
	\end{itemize}




\end{enumerate}






\end{document}



